% Paper template for TAR 2022
% (C) 2014 Jan Šnajder, Goran Glavaš, Domagoj Alagić, Mladen Karan
% TakeLab, FER

\documentclass[10pt, a4paper]{article}

\usepackage{tar2023}

\usepackage[utf8]{inputenc}
\usepackage[pdftex]{graphicx}
\usepackage{booktabs}
\usepackage{amsmath}
\usepackage{amssymb}

\title{Unjust Equivalence: Are Irony and Sarcasm Truly the Same in NLP?}

\name{Bojan Puvača, Florijan Sandalj, Ivan Unković} 

\address{
University of Zagreb, Faculty of Electrical Engineering and Computing\\
Unska 3, 10000 Zagreb, Croatia\\ 
\texttt{\{bojan.puvaca, florijan.sandalj, ivan.unkovic\}@fer.hr}\\
}
          
         
\abstract{ 
}

\begin{document}

\maketitleabstract

\section{Introduction}

\section{Irony and sarcasm in NLP}
The relationship between irony and sarcasm is unfortunately a heavily contested subject in NLP. This problem is the easiest
to notice when looking at different sarcasm and irony datasets, where we can find cases when they are treated as completely
seperate concepts \citep{kaggle-tweets}, when sarcasm is treated as a subset of irony \citep{semeval-2018} or even vice
versa \citep{iSarcasm}. Searching for a consensus in the realm of linguistics is a futile effort as well, however we have 
found two distinctions between the two that have merit in the context of NLP.

\subsection{Sarcasm - irony's meaner cousin}
The online Merriam-Webster dictionary defines sarcasm as "a sharp and often satirical or ironic utterance designed to 
cut or give pain" \citep{mw-dictionary}. This definition seems to be in line with the general consensus that sarcasm is
a form of irony that is more aggressive and mean-spirited. The iSarcasm dataset \citep{iSarcasm} is a good example of this 
categorization, as the "sarcasm" label is, in fact, a subset of the unfortunately named "sarcastic" label, which actually 
indicates any kind of ironic speech. 
In this context, irony refers to any type of speech that is based on polarity -
expressing with words the opposite of what we mean. Although this definition works on paper, there are some pitfalls.
Most notably, the line between sarcasm and irony is unclear, as whether or not a statement is mean-spirited is subjective.
Also, tweets that contain irony and aren't directed at a specific person can still be considered sarcastic, as they often
target a group of people, concepts, ideas or themselves in the form of self-deprecating humor. How the object of the irony
affects the classification is a question that remains unanswered.
All things considered, this distinction is a solid starting point for NLP research, and the iSarcasm \citep{iSarcasm} dataset
does a solid job at distinguishing between the two. However, the usefullness of this distinction is limited, as both concepts
are based on dishonest speech, meaning that in practice there isn't much use in distinguishing between the two.

\subsection{Sarcasm - the figure of speech}
In search of a 

\section{Experimental setup}
\subsection{Sarcasm detection dataset}
\citep{iSarcasm}
\section{Results}

\section{Discussion}

\section{Conclusion}

\section*{Acknowledgements}

\bibliographystyle{tar2023}
\bibliography{tar2023} 

\end{document}

