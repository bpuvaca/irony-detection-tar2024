% Paper template for TAR 2022
% (C) 2014 Jan Šnajder, Goran Glavaš, Domagoj Alagić, Mladen Karan
% TakeLab, FER

\documentclass[10pt, a4paper]{article}

\usepackage{tar2023}

\usepackage[utf8]{inputenc}
\usepackage[pdftex]{graphicx}
\usepackage{booktabs}
\usepackage{amsmath}
\usepackage{amssymb}

\title{Unjust Equivalence: Are Irony and Sarcasm Truly the Same in NLP?}

\name{Florijan Sandalj, Bojan Puvača, Ivan Unković} 

\address{
University of Zagreb, Faculty of Electrical Engineering and Computing\\
Unska 3, 10000 Zagreb, Croatia\\ 
\texttt{\{florijan.sandalj, bojan.puvaca, ivan.unkovic\}@fer.hr}\\
}
          
         
\abstract{ 
}

\begin{document}

\maketitleabstract

\section{Introduction}

\section{Irony and sarcasm in NLP}
The relationship between irony and sarcasm is unfortunately a heavily contested subject in NLP. This problem is the easiest to see when looking at different sarcasm and irony datasets, where
we can find cases when they are treated as completely seperate concepts \citep{kaggle-tweets}, when sarcasm is treated as a subset of irony \citep{semeval-2018} or even vice versa \citep{iSarcasm}
Searching for the definite answer in the realm of linguistics is a futile effort as well, however we have found two distinctions between the two that have merit in the context of NLP.

\subsection{Sarcasm - irony's meaner cousin}
The online Merriam-Webster dictionary defines sarcasm as "a sharp and often satirical or ironic utterance designed to cut or give pain" \citep{mw-dictionary}. This definition seems to be in line
with the general consensus that sarcasm is a form of irony that is more aggressive and mean-spirited. The iSarcasm dataset \citep{iSarcasm} is a good example of this categorization, as the "sarcasm"
label is, in fact, a subset of the unfortunately named "sarcastic" label, which actually indicates any kind of ironic speech.

\subsection{Sarcasm - the figure of speech}

\section{Experimental setup}
\subsection{Sarcasm detection dataset}
\citep{iSarcasm}
\section{Results}

\section{Discussion}

\section{Conclusion}

\section*{Acknowledgements}

\bibliographystyle{tar2023}
\bibliography{tar2023} 

\end{document}

